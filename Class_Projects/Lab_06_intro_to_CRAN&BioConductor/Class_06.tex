% Options for packages loaded elsewhere
\PassOptionsToPackage{unicode}{hyperref}
\PassOptionsToPackage{hyphens}{url}
\PassOptionsToPackage{dvipsnames,svgnames,x11names}{xcolor}
%
\documentclass[
  letterpaper,
  DIV=11,
  numbers=noendperiod]{scrartcl}

\usepackage{amsmath,amssymb}
\usepackage{lmodern}
\usepackage{iftex}
\ifPDFTeX
  \usepackage[T1]{fontenc}
  \usepackage[utf8]{inputenc}
  \usepackage{textcomp} % provide euro and other symbols
\else % if luatex or xetex
  \usepackage{unicode-math}
  \defaultfontfeatures{Scale=MatchLowercase}
  \defaultfontfeatures[\rmfamily]{Ligatures=TeX,Scale=1}
\fi
% Use upquote if available, for straight quotes in verbatim environments
\IfFileExists{upquote.sty}{\usepackage{upquote}}{}
\IfFileExists{microtype.sty}{% use microtype if available
  \usepackage[]{microtype}
  \UseMicrotypeSet[protrusion]{basicmath} % disable protrusion for tt fonts
}{}
\makeatletter
\@ifundefined{KOMAClassName}{% if non-KOMA class
  \IfFileExists{parskip.sty}{%
    \usepackage{parskip}
  }{% else
    \setlength{\parindent}{0pt}
    \setlength{\parskip}{6pt plus 2pt minus 1pt}}
}{% if KOMA class
  \KOMAoptions{parskip=half}}
\makeatother
\usepackage{xcolor}
\setlength{\emergencystretch}{3em} % prevent overfull lines
\setcounter{secnumdepth}{-\maxdimen} % remove section numbering
% Make \paragraph and \subparagraph free-standing
\ifx\paragraph\undefined\else
  \let\oldparagraph\paragraph
  \renewcommand{\paragraph}[1]{\oldparagraph{#1}\mbox{}}
\fi
\ifx\subparagraph\undefined\else
  \let\oldsubparagraph\subparagraph
  \renewcommand{\subparagraph}[1]{\oldsubparagraph{#1}\mbox{}}
\fi

\usepackage{color}
\usepackage{fancyvrb}
\newcommand{\VerbBar}{|}
\newcommand{\VERB}{\Verb[commandchars=\\\{\}]}
\DefineVerbatimEnvironment{Highlighting}{Verbatim}{commandchars=\\\{\}}
% Add ',fontsize=\small' for more characters per line
\usepackage{framed}
\definecolor{shadecolor}{RGB}{241,243,245}
\newenvironment{Shaded}{\begin{snugshade}}{\end{snugshade}}
\newcommand{\AlertTok}[1]{\textcolor[rgb]{0.68,0.00,0.00}{#1}}
\newcommand{\AnnotationTok}[1]{\textcolor[rgb]{0.37,0.37,0.37}{#1}}
\newcommand{\AttributeTok}[1]{\textcolor[rgb]{0.40,0.45,0.13}{#1}}
\newcommand{\BaseNTok}[1]{\textcolor[rgb]{0.68,0.00,0.00}{#1}}
\newcommand{\BuiltInTok}[1]{\textcolor[rgb]{0.00,0.23,0.31}{#1}}
\newcommand{\CharTok}[1]{\textcolor[rgb]{0.13,0.47,0.30}{#1}}
\newcommand{\CommentTok}[1]{\textcolor[rgb]{0.37,0.37,0.37}{#1}}
\newcommand{\CommentVarTok}[1]{\textcolor[rgb]{0.37,0.37,0.37}{\textit{#1}}}
\newcommand{\ConstantTok}[1]{\textcolor[rgb]{0.56,0.35,0.01}{#1}}
\newcommand{\ControlFlowTok}[1]{\textcolor[rgb]{0.00,0.23,0.31}{#1}}
\newcommand{\DataTypeTok}[1]{\textcolor[rgb]{0.68,0.00,0.00}{#1}}
\newcommand{\DecValTok}[1]{\textcolor[rgb]{0.68,0.00,0.00}{#1}}
\newcommand{\DocumentationTok}[1]{\textcolor[rgb]{0.37,0.37,0.37}{\textit{#1}}}
\newcommand{\ErrorTok}[1]{\textcolor[rgb]{0.68,0.00,0.00}{#1}}
\newcommand{\ExtensionTok}[1]{\textcolor[rgb]{0.00,0.23,0.31}{#1}}
\newcommand{\FloatTok}[1]{\textcolor[rgb]{0.68,0.00,0.00}{#1}}
\newcommand{\FunctionTok}[1]{\textcolor[rgb]{0.28,0.35,0.67}{#1}}
\newcommand{\ImportTok}[1]{\textcolor[rgb]{0.00,0.46,0.62}{#1}}
\newcommand{\InformationTok}[1]{\textcolor[rgb]{0.37,0.37,0.37}{#1}}
\newcommand{\KeywordTok}[1]{\textcolor[rgb]{0.00,0.23,0.31}{#1}}
\newcommand{\NormalTok}[1]{\textcolor[rgb]{0.00,0.23,0.31}{#1}}
\newcommand{\OperatorTok}[1]{\textcolor[rgb]{0.37,0.37,0.37}{#1}}
\newcommand{\OtherTok}[1]{\textcolor[rgb]{0.00,0.23,0.31}{#1}}
\newcommand{\PreprocessorTok}[1]{\textcolor[rgb]{0.68,0.00,0.00}{#1}}
\newcommand{\RegionMarkerTok}[1]{\textcolor[rgb]{0.00,0.23,0.31}{#1}}
\newcommand{\SpecialCharTok}[1]{\textcolor[rgb]{0.37,0.37,0.37}{#1}}
\newcommand{\SpecialStringTok}[1]{\textcolor[rgb]{0.13,0.47,0.30}{#1}}
\newcommand{\StringTok}[1]{\textcolor[rgb]{0.13,0.47,0.30}{#1}}
\newcommand{\VariableTok}[1]{\textcolor[rgb]{0.07,0.07,0.07}{#1}}
\newcommand{\VerbatimStringTok}[1]{\textcolor[rgb]{0.13,0.47,0.30}{#1}}
\newcommand{\WarningTok}[1]{\textcolor[rgb]{0.37,0.37,0.37}{\textit{#1}}}

\providecommand{\tightlist}{%
  \setlength{\itemsep}{0pt}\setlength{\parskip}{0pt}}\usepackage{longtable,booktabs,array}
\usepackage{calc} % for calculating minipage widths
% Correct order of tables after \paragraph or \subparagraph
\usepackage{etoolbox}
\makeatletter
\patchcmd\longtable{\par}{\if@noskipsec\mbox{}\fi\par}{}{}
\makeatother
% Allow footnotes in longtable head/foot
\IfFileExists{footnotehyper.sty}{\usepackage{footnotehyper}}{\usepackage{footnote}}
\makesavenoteenv{longtable}
\usepackage{graphicx}
\makeatletter
\def\maxwidth{\ifdim\Gin@nat@width>\linewidth\linewidth\else\Gin@nat@width\fi}
\def\maxheight{\ifdim\Gin@nat@height>\textheight\textheight\else\Gin@nat@height\fi}
\makeatother
% Scale images if necessary, so that they will not overflow the page
% margins by default, and it is still possible to overwrite the defaults
% using explicit options in \includegraphics[width, height, ...]{}
\setkeys{Gin}{width=\maxwidth,height=\maxheight,keepaspectratio}
% Set default figure placement to htbp
\makeatletter
\def\fps@figure{htbp}
\makeatother

\KOMAoption{captions}{tableheading}
\makeatletter
\makeatother
\makeatletter
\makeatother
\makeatletter
\@ifpackageloaded{caption}{}{\usepackage{caption}}
\AtBeginDocument{%
\ifdefined\contentsname
  \renewcommand*\contentsname{Table of contents}
\else
  \newcommand\contentsname{Table of contents}
\fi
\ifdefined\listfigurename
  \renewcommand*\listfigurename{List of Figures}
\else
  \newcommand\listfigurename{List of Figures}
\fi
\ifdefined\listtablename
  \renewcommand*\listtablename{List of Tables}
\else
  \newcommand\listtablename{List of Tables}
\fi
\ifdefined\figurename
  \renewcommand*\figurename{Figure}
\else
  \newcommand\figurename{Figure}
\fi
\ifdefined\tablename
  \renewcommand*\tablename{Table}
\else
  \newcommand\tablename{Table}
\fi
}
\@ifpackageloaded{float}{}{\usepackage{float}}
\floatstyle{ruled}
\@ifundefined{c@chapter}{\newfloat{codelisting}{h}{lop}}{\newfloat{codelisting}{h}{lop}[chapter]}
\floatname{codelisting}{Listing}
\newcommand*\listoflistings{\listof{codelisting}{List of Listings}}
\makeatother
\makeatletter
\@ifpackageloaded{caption}{}{\usepackage{caption}}
\@ifpackageloaded{subcaption}{}{\usepackage{subcaption}}
\makeatother
\makeatletter
\@ifpackageloaded{tcolorbox}{}{\usepackage[many]{tcolorbox}}
\makeatother
\makeatletter
\@ifundefined{shadecolor}{\definecolor{shadecolor}{rgb}{.97, .97, .97}}
\makeatother
\makeatletter
\makeatother
\ifLuaTeX
  \usepackage{selnolig}  % disable illegal ligatures
\fi
\IfFileExists{bookmark.sty}{\usepackage{bookmark}}{\usepackage{hyperref}}
\IfFileExists{xurl.sty}{\usepackage{xurl}}{} % add URL line breaks if available
\urlstyle{same} % disable monospaced font for URLs
\hypersetup{
  pdftitle={LAB\_06},
  pdfauthor={RUNQI ZHANG},
  colorlinks=true,
  linkcolor={blue},
  filecolor={Maroon},
  citecolor={Blue},
  urlcolor={Blue},
  pdfcreator={LaTeX via pandoc}}

\title{LAB\_06}
\author{RUNQI ZHANG}
\date{}

\begin{document}
\maketitle
\ifdefined\Shaded\renewenvironment{Shaded}{\begin{tcolorbox}[enhanced, sharp corners, borderline west={3pt}{0pt}{shadecolor}, boxrule=0pt, frame hidden, breakable, interior hidden]}{\end{tcolorbox}}\fi

\renewcommand*\contentsname{Table of contents}
{
\hypersetup{linkcolor=}
\setcounter{tocdepth}{3}
\tableofcontents
}
\hypertarget{example-input-vectors-to-start-with}{%
\section{Example input vectors to start
with}\label{example-input-vectors-to-start-with}}

\begin{Shaded}
\begin{Highlighting}[]
\NormalTok{student1 }\OtherTok{\textless{}{-}} \FunctionTok{c}\NormalTok{(}\DecValTok{100}\NormalTok{, }\DecValTok{100}\NormalTok{, }\DecValTok{100}\NormalTok{, }\DecValTok{100}\NormalTok{, }\DecValTok{100}\NormalTok{, }\DecValTok{100}\NormalTok{, }\DecValTok{100}\NormalTok{, }\DecValTok{90}\NormalTok{)}
\NormalTok{student2 }\OtherTok{\textless{}{-}} \FunctionTok{c}\NormalTok{(}\DecValTok{100}\NormalTok{, }\ConstantTok{NA}\NormalTok{, }\DecValTok{90}\NormalTok{, }\DecValTok{90}\NormalTok{, }\DecValTok{90}\NormalTok{, }\DecValTok{90}\NormalTok{, }\DecValTok{97}\NormalTok{, }\DecValTok{80}\NormalTok{)}
\NormalTok{student3 }\OtherTok{\textless{}{-}} \FunctionTok{c}\NormalTok{(}\DecValTok{90}\NormalTok{, }\ConstantTok{NA}\NormalTok{, }\ConstantTok{NA}\NormalTok{, }\ConstantTok{NA}\NormalTok{, }\ConstantTok{NA}\NormalTok{, }\ConstantTok{NA}\NormalTok{, }\ConstantTok{NA}\NormalTok{, }\ConstantTok{NA}\NormalTok{)}
\end{Highlighting}
\end{Shaded}

``student\_homework'' import

\begin{Shaded}
\begin{Highlighting}[]
\NormalTok{student\_homework }\OtherTok{=} \FunctionTok{read.csv}\NormalTok{(}\StringTok{"C:}\SpecialCharTok{\textbackslash{}\textbackslash{}}\StringTok{Users}\SpecialCharTok{\textbackslash{}\textbackslash{}}\StringTok{zhang}\SpecialCharTok{\textbackslash{}\textbackslash{}}\StringTok{OneDrive}\SpecialCharTok{\textbackslash{}\textbackslash{}}\StringTok{桌面}\SpecialCharTok{\textbackslash{}\textbackslash{}}\StringTok{BIMM 143 {-} Bioinformatics Lab}\SpecialCharTok{\textbackslash{}\textbackslash{}}\StringTok{Lab\_06}\SpecialCharTok{\textbackslash{}\textbackslash{}}\StringTok{class\_06}\SpecialCharTok{\textbackslash{}\textbackslash{}}\StringTok{student\_homework.csv"}\NormalTok{)}
\FunctionTok{View}\NormalTok{(student\_homework)}
\FunctionTok{nrow}\NormalTok{(student\_homework)}
\end{Highlighting}
\end{Shaded}

\begin{verbatim}
[1] 20
\end{verbatim}

\begin{Shaded}
\begin{Highlighting}[]
\FunctionTok{ncol}\NormalTok{(student\_homework)}
\end{Highlighting}
\end{Shaded}

\begin{verbatim}
[1] 6
\end{verbatim}

\hypertarget{practice}{%
\section{Practice}\label{practice}}

package import

\begin{Shaded}
\begin{Highlighting}[]
\FunctionTok{library}\NormalTok{(FSA)}
\end{Highlighting}
\end{Shaded}

\begin{verbatim}
## FSA v0.9.3. See citation('FSA') if used in publication.
## Run fishR() for related website and fishR('IFAR') for related book.
\end{verbatim}

\begin{Shaded}
\begin{Highlighting}[]
\FunctionTok{library}\NormalTok{(tidyverse)}
\end{Highlighting}
\end{Shaded}

\begin{verbatim}
-- Attaching packages --------------------------------------- tidyverse 1.3.2 --
\end{verbatim}

\begin{verbatim}
v ggplot2 3.3.6      v purrr   0.3.5 
v tibble  3.1.8      v dplyr   1.0.10
v tidyr   1.2.1      v stringr 1.4.1 
v readr   2.1.3      v forcats 0.5.2 
-- Conflicts ------------------------------------------ tidyverse_conflicts() --
x dplyr::filter() masks stats::filter()
x dplyr::lag()    masks stats::lag()
\end{verbatim}

\begin{Shaded}
\begin{Highlighting}[]
\FunctionTok{library}\NormalTok{(ggplot2)}
\FunctionTok{library}\NormalTok{(}\StringTok{"gridExtra"}\NormalTok{)}
\end{Highlighting}
\end{Shaded}

\begin{verbatim}

Attaching package: 'gridExtra'

The following object is masked from 'package:dplyr':

    combine
\end{verbatim}

replace NA with 0

\begin{Shaded}
\begin{Highlighting}[]
\NormalTok{student\_homework[}\FunctionTok{is.na}\NormalTok{(student\_homework)] }\OtherTok{\textless{}{-}} \DecValTok{0}
\end{Highlighting}
\end{Shaded}

calculate the sum and min for each student

\begin{Shaded}
\begin{Highlighting}[]
\NormalTok{student\_min }\OtherTok{\textless{}{-}} \FunctionTok{apply}\NormalTok{(student\_homework[,}\SpecialCharTok{{-}}\DecValTok{1}\NormalTok{], }\DecValTok{1}\NormalTok{, min)}
\NormalTok{student\_min}
\end{Highlighting}
\end{Shaded}

\begin{verbatim}
 [1] 73 64 69  0 75 77 74 76 77  0 66 70 76 76  0 74 63  0 68 68
\end{verbatim}

\begin{Shaded}
\begin{Highlighting}[]
\NormalTok{student\_sum }\OtherTok{\textless{}{-}} \FunctionTok{apply}\NormalTok{(student\_homework[,}\SpecialCharTok{{-}}\DecValTok{1}\NormalTok{], }\DecValTok{1}\NormalTok{, sum)}
\NormalTok{student\_sum}
\end{Highlighting}
\end{Shaded}

\begin{verbatim}
 [1] 440 394 406 337 428 433 450 451 428 316 410 437 445 427 315 432 415 378 399
[20] 399
\end{verbatim}

\hypertarget{substract-the-total-score-by-their-lowest-score}{%
\section{substract the total score by their lowest
score}\label{substract-the-total-score-by-their-lowest-score}}

\begin{Shaded}
\begin{Highlighting}[]
\NormalTok{adjusted\_sum }\OtherTok{\textless{}{-}}\NormalTok{ student\_sum }\SpecialCharTok{{-}}\NormalTok{ student\_min}
\NormalTok{adjusted\_sum}
\end{Highlighting}
\end{Shaded}

\begin{verbatim}
 [1] 367 330 337 337 353 356 376 375 351 316 344 367 369 351 315 358 352 378 331
[20] 331
\end{verbatim}

\#Average = total/ \#assignment one col contains sutdent name, one col
is disposed, \#remaining scores = ncol-2

\begin{Shaded}
\begin{Highlighting}[]
\NormalTok{student\_mean }\OtherTok{\textless{}{-}}\NormalTok{ adjusted\_sum}\SpecialCharTok{/}\NormalTok{(}\FunctionTok{ncol}\NormalTok{(student\_homework)}\SpecialCharTok{{-}}\DecValTok{2}\NormalTok{)}
\NormalTok{student\_mean}
\end{Highlighting}
\end{Shaded}

\begin{verbatim}
 [1] 91.75 82.50 84.25 84.25 88.25 89.00 94.00 93.75 87.75 79.00 86.00 91.75
[13] 92.25 87.75 78.75 89.50 88.00 94.50 82.75 82.75
\end{verbatim}

\#transpose col to row and append it to the original csv document

\begin{Shaded}
\begin{Highlighting}[]
\NormalTok{student\_mean }\OtherTok{\textless{}{-}} \FunctionTok{t}\NormalTok{(}\FunctionTok{t}\NormalTok{(student\_mean))}
\FunctionTok{nrow}\NormalTok{(student\_mean)}
\end{Highlighting}
\end{Shaded}

\begin{verbatim}
[1] 20
\end{verbatim}

\begin{Shaded}
\begin{Highlighting}[]
\NormalTok{student\_final }\OtherTok{\textless{}{-}} \FunctionTok{cbind}\NormalTok{(student\_homework, student\_mean)}
\end{Highlighting}
\end{Shaded}

\hypertarget{q1}{%
\section{Q1:}\label{q1}}

\textbf{\#integrate the codes above into a R function named grade()} 1.
takes into a parameter, namely the gradebook 2. replace NA with 0 3.
calculate the total and min for each student 4. calculate the adjusted
score, substract total by min 5. calculate the final score (average),
divide adjusted total by n-2

\begin{Shaded}
\begin{Highlighting}[]
\NormalTok{grade }\OtherTok{\textless{}{-}} \ControlFlowTok{function}\NormalTok{(grade\_raw) \{}
\NormalTok{   grade\_raw[}\FunctionTok{is.na}\NormalTok{(grade\_raw)] }\OtherTok{\textless{}{-}} \DecValTok{0} \CommentTok{\#replace NA with numerical 0}
   
   \CommentTok{\#student\textquotesingle{}s sum subtracted by min, and transposed into rows}
\NormalTok{   grade\_adjusted }\OtherTok{=} \FunctionTok{t}\NormalTok{( }\FunctionTok{t}\NormalTok{( }\FunctionTok{apply}\NormalTok{(grade\_raw[,}\SpecialCharTok{{-}}\DecValTok{1}\NormalTok{], }\DecValTok{1}\NormalTok{, sum) }\SpecialCharTok{{-}} \FunctionTok{apply}\NormalTok{(grade\_raw[,}\SpecialCharTok{{-}}\DecValTok{1}\NormalTok{], }\DecValTok{1}\NormalTok{, min) ) ) }
   
\NormalTok{   grade\_mean }\OtherTok{=}\NormalTok{ grade\_adjusted }\SpecialCharTok{/}\NormalTok{ (}\FunctionTok{ncol}\NormalTok{(grade\_raw)}\SpecialCharTok{{-}}\DecValTok{2}\NormalTok{) }\CommentTok{\#calculate the mean}
\NormalTok{   grade\_final }\OtherTok{=} \FunctionTok{cbind}\NormalTok{(grade\_raw, grade\_mean) }\CommentTok{\#integrate the mean into a final document}
   \FunctionTok{return}\NormalTok{(grade\_final) }
\NormalTok{\}}
\end{Highlighting}
\end{Shaded}

\#test run

\begin{Shaded}
\begin{Highlighting}[]
\NormalTok{test\_doc}\OtherTok{=} \FunctionTok{read.csv}\NormalTok{(}\StringTok{"C:}\SpecialCharTok{\textbackslash{}\textbackslash{}}\StringTok{Users}\SpecialCharTok{\textbackslash{}\textbackslash{}}\StringTok{zhang}\SpecialCharTok{\textbackslash{}\textbackslash{}}\StringTok{OneDrive}\SpecialCharTok{\textbackslash{}\textbackslash{}}\StringTok{桌面}\SpecialCharTok{\textbackslash{}\textbackslash{}}\StringTok{BIMM 143 {-} Bioinformatics Lab}\SpecialCharTok{\textbackslash{}\textbackslash{}}\StringTok{Lab\_06}\SpecialCharTok{\textbackslash{}\textbackslash{}}\StringTok{class\_06}\SpecialCharTok{\textbackslash{}\textbackslash{}}\StringTok{student\_homework.csv"}\NormalTok{)}
\FunctionTok{grade}\NormalTok{(test\_doc)}
\end{Highlighting}
\end{Shaded}

\begin{verbatim}
            X hw1 hw2 hw3 hw4 hw5 grade_mean
1   student-1 100  73 100  88  79      91.75
2   student-2  85  64  78  89  78      82.50
3   student-3  83  69  77 100  77      84.25
4   student-4  88   0  73 100  76      84.25
5   student-5  88 100  75  86  79      88.25
6   student-6  89  78 100  89  77      89.00
7   student-7  89 100  74  87 100      94.00
8   student-8  89 100  76  86 100      93.75
9   student-9  86 100  77  88  77      87.75
10 student-10  89  72  79   0  76      79.00
11 student-11  82  66  78  84 100      86.00
12 student-12 100  70  75  92 100      91.75
13 student-13  89 100  76 100  80      92.25
14 student-14  85 100  77  89  76      87.75
15 student-15  85  65  76  89   0      78.75
16 student-16  92 100  74  89  77      89.50
17 student-17  88  63 100  86  78      88.00
18 student-18  91   0 100  87 100      94.50
19 student-19  91  68  75  86  79      82.75
20 student-20  91  68  76  88  76      82.75
\end{verbatim}

\hypertarget{q2-using-your-grade-function-and-the-supplied-gradebook-who-is-the-top-scoring-student}{%
\section{Q2: Using your grade() function and the supplied gradebook, Who
is the top scoring
student}\label{q2-using-your-grade-function-and-the-supplied-gradebook-who-is-the-top-scoring-student}}

overall in the gradebook?

\textbf{A2: student-18 is the top socring student with an final score of
94.5}

\begin{Shaded}
\begin{Highlighting}[]
\FunctionTok{max}\NormalTok{(}\FunctionTok{grade}\NormalTok{(student\_homework)}\SpecialCharTok{$}\NormalTok{grade\_mean)}
\end{Highlighting}
\end{Shaded}

\begin{verbatim}
[1] 94.5
\end{verbatim}

\begin{Shaded}
\begin{Highlighting}[]
\FunctionTok{which.max}\NormalTok{(}\FunctionTok{grade}\NormalTok{(student\_homework)}\SpecialCharTok{$}\NormalTok{grade\_mean)}
\end{Highlighting}
\end{Shaded}

\begin{verbatim}
[1] 18
\end{verbatim}

\hypertarget{q3-from-your-analysis-of-the-gradebook-which-homework-was-toughest-on-students-i.e.-obtained-the-lowest-scores-overall}{%
\section{Q3: From your analysis of the gradebook, which homework was
toughest on students (i.e.~obtained the lowest scores
overall?}\label{q3-from-your-analysis-of-the-gradebook-which-homework-was-toughest-on-students-i.e.-obtained-the-lowest-scores-overall}}

\textbf{A3: homework\#2 is the toughest because it has the lowest
average and the lowest median}

\hypertarget{q4-from-your-analysis-of-the-gradebook-which-homework-was-most-predictive-of-overall-score-i.e.-highest-correlation-with-average-grade-score}{%
\section{Q4: From your analysis of the gradebook, which homework was
most predictive of overall score (i.e.~highest correlation with average
grade
score)?}\label{q4-from-your-analysis-of-the-gradebook-which-homework-was-most-predictive-of-overall-score-i.e.-highest-correlation-with-average-grade-score}}

\textbf{A4: HW5 was most predictive of final score because it shows the
higest correlation.}

\begin{Shaded}
\begin{Highlighting}[]
\NormalTok{student\_final }\OtherTok{=} \FunctionTok{grade}\NormalTok{(student\_homework)}
\FunctionTok{cor}\NormalTok{(student\_final}\SpecialCharTok{$}\NormalTok{grade\_mean, student\_final[}\DecValTok{2}\SpecialCharTok{:}\DecValTok{6}\NormalTok{])}
\end{Highlighting}
\end{Shaded}

\begin{verbatim}
           hw1      hw2       hw3       hw4       hw5
[1,] 0.4250204 0.176778 0.3042561 0.3810884 0.6325982
\end{verbatim}

\begin{Shaded}
\begin{Highlighting}[]
\CommentTok{\#alternatively}
\FunctionTok{apply}\NormalTok{(student\_homework[,}\SpecialCharTok{{-}}\DecValTok{1}\NormalTok{], }\DecValTok{2}\NormalTok{, cor, }\AttributeTok{y=}\NormalTok{student\_final}\SpecialCharTok{$}\NormalTok{grade\_mean)}
\end{Highlighting}
\end{Shaded}

\begin{verbatim}
      hw1       hw2       hw3       hw4       hw5 
0.4250204 0.1767780 0.3042561 0.3810884 0.6325982 
\end{verbatim}

\hypertarget{q5-make-sure-you-save-your-quarto-document-and-can-click-the-render-or-rmarkdownknit-button-to-generate-a-pdf-foramt-report-without-errors.-finally-submit-your-pdf}{%
\section{Q5: Make sure you save your Quarto document and can click the
``Render'' (or Rmarkdown''Knit'') button to generate a PDF foramt report
without errors. Finally, submit your
PDF}\label{q5-make-sure-you-save-your-quarto-document-and-can-click-the-render-or-rmarkdownknit-button-to-generate-a-pdf-foramt-report-without-errors.-finally-submit-your-pdf}}

to gradescope.



\end{document}
